\documentclass{beamer}

% Theme selection - 'Madrid' is clean and professional for academic use
\usetheme{Madrid}
\usecolortheme{beaver} % A nice red/grey theme often used in academia

% Metadata
\title{Introduction to Python}
\subtitle{CMPT 211: Winter 2026}
\author{Bernal Manzanilla}
\institute{Concordia University of Edmonton}
\date{\today}

\begin{document}

% Title Slide
\begin{frame}
    \titlepage
\end{frame}

% Slide 1: What is Python?
\begin{frame}{What is Python?}
    \begin{itemize}
        \item \textbf{High-level Language:} Python abstracts away complex details of the computer (memory management, pointers).
        \item \textbf{Interpreted Language:}
        \begin{itemize}
            \item Unlike C or C++, Python code is not compiled directly to machine code before execution.
            \item It is translated into \textit{bytecode} which is then executed by the Python Virtual Machine (PVM).
            \item This allows for rapid development and interactivity: Read–Eval–Print Loop (REPL).
        \end{itemize}
    \end{itemize}
\end{frame}

% Slide 2: The "Need for Speed" (Numerical Backends)
\begin{frame}{Performance \& Numerical Computing}
    \begin{block}{The "Glue" Language}
        Python is famous for being excellent "glue" code. It connects different software components effortlessly.
    \end{block}
    
    \vspace{0.5cm}
    
    \textbf{Under the Hood:}
    \begin{itemize}
        \item While pure Python can be slower than C++, libraries like \textbf{NumPy} and \textbf{SciPy} bypass this limitation.
        \item These libraries run highly optimized numerical routines written in \textbf{C} and \textbf{FORTRAN}.
        \item \textbf{Result:} You get the ease of Python with the speed of compiled languages for heavy math operations.
    \end{itemize}
\end{frame}

% Slide 3: Is it Just-In-Time (JIT)?
\begin{frame}{Is Python JIT Compiled?}
    \begin{alertblock}{Standard Python (CPython)}
        The standard version of Python you install is \textbf{not} JIT compiled. It is an interpreter.
    \end{alertblock}
    
    \vspace{0.5cm}
    
    \textbf{When is JIT used?}
    \begin{itemize}
        \item \textbf{PyPy:} An alternative implementation of Python that uses a JIT compiler to run faster.
        \item \textbf{Numba:} A library that allows you to JIT compile specific functions in your code to machine code (e.g., using decorators like \texttt{@jit}).
    \end{itemize}
\end{frame}

% Slide 4: Programming Paradigms
\begin{frame}{Two Main Paradigms}
    Python supports multiple ways to structure code. We will focus on two:
    
    \vspace{0.5cm}
    
    \begin{enumerate}
        \item \textbf{Structured Programming (Imperative)}
        \begin{itemize}
            \item The code is a sequence of instructions.
            \item Focuses on control flow: loops, conditions, and functions.
            \item \textit{Note: This is the primary paradigm we will use to build logic in this course.}
        \end{itemize}
        
        \vspace{0.3cm}
        
        \item \textbf{Object-Oriented Programming (OOP)}
        \begin{itemize}
            \item Data and functions are bundled together into "Objects".
            \item Useful for large systems, but adds complexity.
        \end{itemize}
    \end{enumerate}
\end{frame}

\end{document}